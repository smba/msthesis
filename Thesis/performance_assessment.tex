\section{Statistical Considerations}\label{sec:statistical_considerations}
In this section, we now discuss the appropriate statistical means to summarize,
compare and interpret measurement results. For the remainder of this section, we
will refer to the following two scenarios. First, to obtain robust results, the
assessment of a single variant needs to be repeated multiple times.
Consequently, to report a single result per metric, the measurements for a
single variant need to be summarized. Second, the assessment of performance for
a variant may comprise several use cases, for instance file
compression and decompression for a compression software. Therefore, performance measures
aggregated from different benchmarks need to be summarized accurately. 

\subsection{Measures of Central Tendency}
For each test run of a software system or variant, we obtain a single-valued
measurement per performance metric. Since we repeat each test run $n$ times
per variant, we obtain a data record $X$ with $n$ measurements $X = X_1, X_2,
\ldots, X_{n-1}, X_n$. While the arithmetic mean is commonly considered the
right way to summarize data records and report a representative average
value, we need to be more cautious with how to summarize data records
\citep{fleming_how_1986,smith_characterizing_1988}. From a statistical
perspective, the intention of summarizing a data record is to find a measure of
central tendency what is representative for the data record.
While the arithmetic mean is an appropriate method for many cases, there exist
other means to summarize data records. Moreover, there exist a number of
criteria for when to use which means to summarize a data record. 

The first question when summarizing a data
record is to ask what the data actually describe and what we intend to express
with our summary. For a data record $X$, we can define a
relationship we would like to conserve while replacing each single $X_i$ with an
average value $\bar{x}$. Based on this relationship, we can derive the
appropriate method to summarize our record. For instance, our data record $X$
describes the time elapsed for a test case and we want the keep the following
relation, saying that the sum of all measurements $\sum_{i = 1}^{n} X_i$ is
equal to the total time elapsed $T$, defined as

$$
T = \sum_{i = 1}^{n} X_i = \sum_{i = 1}^{n} \bar{x}.
$$

Based on the term above, we can derive the definition of the method
appropriate to summarize our data record with respect to the conserved relation
what is the \emph{arithmetic mean}, defined as

\begin{equation} \label{eq:arithmetic_mean}
\begin{split}
\bar{x} = \frac{1}{n} \sum_{i = 1}^{n} X_i.
\end{split}
\end{equation}

Consider another example, similar to the one above, where the test
case is a load test with a predefined number of users and the measurements $X =
X_1, X_2, \ldots, X_{n-1}, X_n$ are
measured as hits per second. Again, we have different measurements we want to
summarize with respect to a relation to conserve. Each user drops the same
number of requests $T$, whereby the hit rate $X_i$ and the respective elapsed
time $t_i = \frac{T}{X_i }$ vary. We now conserve the relationship that the
total number of requests for the data record is the sum of all hit rates $X_i$
times the time elapsed $t_i$. Therefore, we want to substitute each hit rate
$X_i$ with an average value $\hat{x}$ so that the aforementioned relationship is
conserved:

$$
n\cdot T - \sum_{n}^{i=1} X_it_i = 0 \Leftrightarrow n\cdot T - \hat{v}
\sum_{n}^{i=1} t_i = 0 \Leftrightarrow n = \hat{x} \sum_{n}^{i=1} \frac{1}{X_i}
$$

Similar to Eq. \ref{eq:arithmetic_mean} we can derive the definition for the summarization
method to use from the equation above what is the \emph{harmonic mean}, defined as

\begin{equation} \label{eq:harmonic_mean}
\begin{split}
\hat{x} = n \cdot \bigg(\sum_{i=1}^{n} \frac{1}{X_i}\bigg)^{-1} =
\frac{n}{\frac{1}{X_1} + \frac{1}{X_2} + \ldots +
\frac{1}{X_{n-1}} + \frac{1}{X_n}}.
\end{split}
\end{equation}

We see that the summarization methods presented in Eq. \ref{eq:arithmetic_mean}
and \ref{eq:harmonic_mean} are useful for different types of data records, and
should be used accordingly. The arithmetic mean is suitable for records, where the total sum of single
measurements has a meaning, whereas the harmonic mean is suitable for measured
rates or frequencies \citep{smith_characterizing_1988}.

While the aforementioned measures of central tendency should be appropriate for
most cases, they are not always the best choice though since the arithmetic
and harmonic mean are heavily influenced by extreme observations
\citep{shanmugam_statistics_2015}. 
For instance, for the data record $X = 1, 2, 3, 30$, three of four values are
smaller than the arithmetic mean $\bar{x} = 9$. One option is to explicitly exclude outlier
values from the data record. The so-called \emph{trimmed mean} is
obtained by truncating a upper and/or lower percentage $t$ of the data record
and, consequently, computing the (arithmetic or harmonic) mean for the remaining
data record \citep{shanmugam_statistics_2015}. 

While this method is suitable to omit the effects of outliers, one still needs to specify
which upper and/or lower percentage $t$ needs to be truncated. Moreover, the use
a (trimmed) mean requires the data records' frequencies to be distributed
symmetrically around its mean.
A probability distribution is skewed (and therefore asymmetric) if, graphically speaking, its histogram is not
symmetric around its measure of central tendency. A simple method to measure the
skewness of a probability distribution is \emph{Bowley's measure}
\citep{shanmugam_statistics_2015}, defined as

\begin{equation} \label{eq:bowley}
\begin{split}
B_S = \frac{(Q_3 - M) - (M - Q_1)}{Q_3 - Q_1} = \frac{(Q_3 + Q_1 - 2M)}{Q_3
- Q_1},
\end{split}
\end{equation}

where $Q_1$ and $Q_3$ denote the first and third quartile, and $M$ denotes the
median of the probability distribution. The quartiles $Q_i$ with $i \in \left\{
1,2,3 \right\}$ are defined as the values of a data record $X$, so that
$\frac{i}{4}$ of the values of $X$ are smaller than $Q_i$. The \emph{median} is
defined as $Q_2$, i.e., a value $M \in X$ so that half of the values in $X$ are smaller than $M$.

The median itself is a more robust measure of central tendency than the
aforementioned ones since it is less influenced by outliers and can be used for
both skewed and symmetric data records \citep{shanmugam_statistics_2015}. For a
given ascendingly ordered data record $X = X_1, X_2, \ldots, X_{n-1}, X_n$, we
can compute the median as follows:

\begin{equation} \label{eq:median}
\mathrm{Median}(X) = \begin{cases}
     X_{\frac{n+1}{2}} & \text{if $n$ is odd} \\
     \frac{1}{2}\big(X_{\frac{n}{2}} + X_{\frac{n}{2}+1}\big) & \text{if $n$ is
     even}
   \end{cases}~\text{with $X_i \leq X_j$ and $i < j \leq n$ }
\end{equation}

\subsection{Measures of Dispersion}

Last, we take a look at different measures of spread or dispersion. Most
commonly used are the \emph{variance} $\sigma^2$ and the \emph{standard
deviation} $\sigma = \sqrt{\sigma^2}$ defined along the mean $\mu$ of a
probability distribution as

\begin{equation} \label{eq:variance}
\begin{split}
\sigma^2 = \frac{\sum_{i=1}^{n}(X_i - \mu)^2}{n}
\end{split}
\end{equation}

Similar to the (arithmetic or harmonic) mean, the variance and the standard
deviation are heavily influenced by extreme observations
\citep{shanmugam_statistics_2015} and not the best choice in all cases. Instead,
two more robust measures are the \emph{median absolute deviation} (MAD) and the \emph{inter-quartile range} (IQR). The MAD is defined as the median of the absolute deviations from the probability distributions'
median \citep{molyneaux_art_2014}, or, defined as follows:

\begin{equation} \label{eq:mad}
\begin{split}
\mathrm{MAD}(X) = \mathrm{Median}\big(|X - \mathrm{Median}(X)|\big)
\end{split}
\end{equation}
 
The IQR, however, defines the range between the first and the third quartile,
$Q_1$ and $Q_3$ \citep{shanmugam_statistics_2015}. This range is larger for a
widespread data record and smaller for a data range with a narrow spread, but is not influenced by extreme
observations as those outliers do not lie within the range $\left[ Q_1,Q_3
\right] $. The IQR is defined as 

\begin{equation} \label{eq:iqr}
\begin{split}
\mathrm{IQR}(X) = Q_3 - Q_1.
\end{split}
\end{equation}

\subsection{When to use which measure?}
In this section we discussed the arithmetic and harmonic mean as well as the
median as a measure of central tendency, the symmetric property that is
required to use the arithmetic or harmonic mean, and different measures of
spread for a given data record. At the beginning, we have raised the question of how to
summarize data records (a) for an experiment repeated multiple times, and (b)
obtained from different benchmarks. While for case (b) the answer is to use the
arithmetic or harmonic mean (depending on the quality of the measurements), for
(a) the answer is a little more elaborate.

The arithmetic (or harmonic) mean can be used whenever the quality of the
measurement is appropriate and the data record is symmetric. According to
\cite{shanmugam_statistics_2015}, the arithmetic mean is preferable ``when the numbers combine
additively to produce a resultant value'', such as time periods or memory sizes, whereas
the harmonic mean is preferable ``when reciprocals of several non-zero numbers
combine additively to produce a resultant value'', such as rates or frequencies
\citep{smith_characterizing_1988}.
The median is a less precise estimator than the both means, yet more robust
with regard to extreme observations. In addition, the MAD or IQR provide a more
robust means of spread than the standard deviation.
