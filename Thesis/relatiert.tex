While to the best of our knowledge, there exists no previous work on
exhaustively presenting a methodology to understand and comprehend performance
evolution of configurable system, existing related work and ours have either
touched on similar topics, or pursue a similar methodology, yet with different
basic premises. In the following we present and summarize related work to
emphasize the limited scope of our work, and to give an outline of possible
future research directions.

\paragraph{Comprehending Software Evolution.} Throughout our methodology, we
have frequently referred to visualizations of data obtained from repository
mining, for instance, for commit activity, and as a basis for revision sampling
strategies. In addition, we have taken into account similar data in the
interpretation of our performance evolution history data in section. More
generally speaking, software as a product is the result of a complex
development process involving a repeatedly revised codebase, testing and
documentation activity as well as records of developer communication, and
organizational artifacts and speifications. As software evolves, hence, it is
inevitable to not consider those data to obtain more complete and integrated
insights. Aggregation and visualization of aforementioned data records can help
to sketch and understand possible coherences.

\cite{german_visualizing_2006} have presented a comprehensive tool,
\emph{softChange}, to visualize multiple aspects of a software system’s
evolution history, including commit and file activity graphs, authorship
overviews, and visualizations of file coupling. While their tool is not
designed with regard to a specific research direction, the authors stress the
importance of means to aggregate and analyze software trails in order to
explore and understand software evolution. Joint visualizations
of software trails as well as performance evolution history data is a promising
augmentation in the comprehension of multi-layered software evolution.

\cite{wu_exploring_2004} have presented presented an integrated approach to
visualize multiple different measurements over time. Their proposed
\emph{evolution spectrum graphs} are inspired by spectrum visualizations for
audio signals. A spectrum in their context can, for instance,  be a list of files, for which
different measurements are illustrated over time. While this approach allows a
global overview over an arbitrary timespan, it also can sketch fine-grained
changes and trends over time. Although, according to the authors, the
visualizations need to be tailored to the target system with respect to
coloring and spectrum selection, evolution spectrum graphs can be a useful
means to aggregate and visualize various single-valued measurements for a
spectrum of variants for configurable software systems.

To conclude with a broader overview on visualization tools for further reading,
\cite{storey_use_2005} have surveyed twelve tools (including the previously
mentioned two) that intend to visualize any kind of human activities in
software development with a string focus on software evolution. The authors
evaluate the different tools with respect to different aspects, including
intent of the tools, the information sources utilized by the tools, the form of
presentation offered by the tools, and the effectiveness in terms of
feasibility and validity.

\paragraph{Power Consumption as a Performance Property.}
With the grown popularity of mobile devices, such as smartphones, over the last
decade, power consumption of software systems has emerged to become a
considerable software quality attribute beside established performance
indicators. In the context of limited resources, end-user requirements
accentuate the demand of power-saving software applications, yet many apps have
net kept up so far \citep{li_empirical_2014}.
 
Recent research activities have taken up this subject, and address the power
consumption of software systems along the lines of research focusing on
software performance engineering. For instance, \cite{sahin_how_2014} have
studied the impact of automated refactorings on the power consumption of almost
200 software systems. The authors’ results indicate that refactorings can both
decrease and increase power consumption, and that a refactorings’ impact on
power consumption is hardly predictable as the effects observed neither
correlated with traditional software performance metrics nor followed any
consistent pattern \citep{sahin_how_2014}.

Moreover, \cite{hindle_green_2015} have proposed a methodology, green mining, to measure
and model the evolution of power consumption across different versions of a
software system. Using their methodology, the authors have documented the power
consumption evolution history of two different software systems, and identified
feasible metrics to summarize power consumption. In contrast to the
observations by \cite{sahin_how_2014}, however, the results by \cite{hindle_green_2015}
suggest a relationship between structural size software metrics and power
consumption. Compared to our proposed methodology, green mining only considers
one-dimensional evolution of power consumption as different software variants
are not taken into account, yet green mining examines the relationship between
power consumption and further quality attributes. That is, we believe that
interpreting performance evolution results in an extended context of software
metrics with respect to software architecture, power consumption, and along
established software analyses should direct further research activities towards
a better understanding of software evolution.

