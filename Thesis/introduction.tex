\paragraph{Configurable Software}
Modern software systems often need to
be customized to satisfy user requirements. Customizable software, for instance,
enables greater flexibility in supporting varying hardware platforms or tweaking
system performance. To make software systems configurable and customizable, they
exhibit a variety of \emph{configuration options}, also called
\emph{features} \citep{apel_feature-oriented_2013}.
Configuration options range from fine-grained options that tune small
functional- and non-functional properties to those that enable or disable entire
parts of the software. The selection of configuration options can be
accommodated at different stages: \emph{compile-} or \emph{build-time} when the
software is built or at \emph{load-time} before the software is actually used.
Compile-time variability usually governs what code sections get
compiled in the program. For instance, compile-time variability can be
realized by excluding code sections from compilation using preprocessor
annotations \citep{hunsen_preprocessor-based_2016} or by assembling the code
sections to compile incrementally from delta modules
\citep{schaefer_delta-oriented_2010}. In contrast to that, load-time variability
controls which code sections can be visited during execution. Configurations for
load-time variability can be specified using configuration files, environment
variables or command-line arguments. Examples for configurable software systems
range from small open-source command-line tools to mature ecosystems including
Eclipse or even operating systems such as the Linux kernel with more than
$11,000$ options \citep{dietrich_robust_2012}.

Configuration options for
software systems are usually constrained (e.g., are mutually exclusive, imply
or depend on other features) to a certain extent. In the worst case though,
where all options can be selected independently, the number of valid
configurations grows exponentially with every feature added and likely exceeds
the number of atoms in the entire universe once we count $265$ independent
features. Hence, even for a small number of features, any naive approach for
assessing emergent properties of configurable software systems exhaustively for
each valid configuration is in general conceived infeasible. Despite this
mathematical limitation, many feasible approaches to static analysis for
configurable systems emerged. Those variability-aware approaches enable, for
instance, type checking in the presence of variability by exploiting
commonalities among different variants \citep{thum_classification_2014}.

To meet functional and non-functional requirements, users aim at finding the
optimal configuration of a configurable software system. However, this task is
non-trivial and has shown to be NP-hard \citep{white_selecting_2009}. The main
driver for the complexity are feature interactions. A \emph{feature
interaction} is an ``emergent behavior that cannot be easily deduced from the
behaviors associated with the individual features involved''
\citep{apel_feature-oriented_2013} and can make development and maintenance of
a configurable system an error-prone task.

To illustrate feature interactions, consider the following example
\citep{siegmund_performance-influence_2015}. A software system, say a file
server, is used to store in a data base and provide access upon request. The
system provides functionality for both encryption and compression. In
isolation, both file en- or decryption and file (de-)compression demand an
expectable fraction of memory and processor time. The performance behavior for
the software system though may vary if both features are selected. For
instance, if a file is encrypted and compressed (or vice versa), we can expect
the operation to demand less resources since an encrypted file is likely to be
of smaller size than the decrypted original.
This is a positive example for a feature interaction, where the performance
behavior although being beneficial is unexpected.

\paragraph{Performance Behavior}
Performance with respect to software and software systems is not
precisely defined and differs from an end user's and a developer’s perspective.
According to \cite{molyneaux_art_2014}, from a user’s perspective ``a well-performing
application is one that lets the end user carry out a given task without any
undue perceived delay or irritation''. However, to accurately assess
performance, from a practitioner’s perspective, performance is outlined by
measurements called \emph{key performance indicators} (KPIs) which relate to
non-functional requirements \citep{molyneaux_art_2014}. The set of KPIs include
availability of a software system, its response time, throughput, and resource
utilization. Availability comprises the amount of time an application is
available to the user. Response time describes the amount of time it takes to
process a task. Throughput describes the program load or number of items passed
to a process. Resource utilization describes the used quota of resources used
for processing a task.

The performance behavior of a software system depends on the functionality
offered, the respective implementation, program load, the underlying hardware system,
environment variables, and the resulting execution. Since configuration options
control what and how functionality is executed, we concentrate here on those
source of performance. While feature interactions not necessarily cause the
software system to break severely in all cases, its overall performance can
become unfavorable for corner cases or specific configurations as the feature
selection influences the execution \citep{foo_mining_2010,heger_automated_2013,nguyen_industrial_2014}. 
That is, the choice of features as well shapes the performance of a software system.

\paragraph{Performance And Evolving Software}
Actively maintained software systems evolve with every modification made, every
version released,  and patch provided. Modifications usually introduce new
functionality to the system, but functionality might as well be divided into
smaller modules to enable provide more fine-grained configuration options. When
features are removed from the software system, the corresponding functionality
might remain in the code base or options are merged
\citep{apel_feature-oriented_2013}.

There exists substantial work on understanding the evolution of configurable
systems, for instance, with respect to software architecture \citep{zhang_variability_2013,passos_feature_2015} or variability
\citep{seidl_co-evolution_2012,peng_analyzing_2011,passos_towards_2012}. As
software evolves, the code base which is subject to modifications and the overall architectural quality can degrade. Common symptoms of architectural degradation are code tangling and scattering
\citep{zhang_variability_2013,passos_feature_2015}, which lead to
less cohesive and stricter coupled code. The more the code base is constrained and interdependent, the more software can become ``brittle''
\citep{perry_software_1991} , less flexible, harder to adapt, and therefore harder to evolve.
Evolution of software, especially with respect to variability, is essentially
driven by and can be conceived as adapting a software system to changed
requirements and contextual changes \citep{peng_analyzing_2011}. That is,
(potential) degradation of software quality as software evolves is often a phenomenon due
to decisions trading quality assurance (QA) or maintenance with meeting
requirements and schedules \citep{guo_tracking_2011}. The metaphor of
\emph{technical debt} \citep{guo_tracking_2011}, which is commonly used to
describe this trade-offs and corresponding costs, outlines the risk that postponed maintenance tasks pose to
software evolution. Although every deferred maintenance or QA task saves some
cost in the first place, it also could also have detected software defects in
the first place. Technical debt implies both interest, so to speak, the
potential damage of a defect depending on its severity, as well as the
probability of incurring interest. A defect can be severe, yet fixable with
reasonable effort and cost. However, the aforementioned symptoms of
architectural degradation and deferring maintenance render bug-fixing to become
more and more expensive.

Besides the aspects of software evolution discussed above, the evolution of
performance for software systems has gained more attention recently. In
practice though, quality assurance with respect to performance is still
conducted to an unsatisfactory extent, or accommodated to late in the
development process, according to \cite{molyneaux_art_2014}. That is, postponed testing
with respect to performance is likely to a driving factor for degradation of
performance quality, or simply called \emph{performance regression}.

While performance discipline have emerged as a discipline of or testing target
in software testing, qualitative root cause analysis, for the most part, is
conducted manually \citep{molyneaux_art_2014}. However,  there exists work on
automated root cause analysis for performance bugs, such as measuring the execution time
of unit tests, whereby an increased execution time indicates performance
regression, and the corresponding unit test helps isolating the root cause
thereof \citep{heger_automated_2013,nguyen_industrial_2014}. In conclusion, we
see that, to better assure good software performance, more knowledge about performance behavior needs to be
available ideally earlier in the development process.

\paragraph{Performance Prediction}
For configurable systems, performance behavior can be more complex and
dependent on the feature selection, as we have seen with the example for
feature interactions above. Similarly, quality assurance for configurable
software systems is far from exhaustively testing all possible configurations,
but rather close to only testing a selection of configurations sampled with
respect to certain constraints. Sampling strategies might stress feature
interactions, such as pair-wise testing, or feature coverage. However, all
sample are selected with the intention to learn as much as possible over the
entire system from a small selection of variants. So to speak, a sampling
strategy is ``optimal'' if for a resulting sample, the probability of missing an
arbitrary software defect, is minimal.

While performance testing is apparently useful, recently a number of techniques
to model and predict performance behavior for arbitrary configurations have
emerged. The idea behind these approaches is similar to sampling strategies as
an ``accurate'' prediction model is able to predict most cases of performance
regressions. The underlying optimization problem of performance prediction
models is to find an accurate estimator for a function describing a performance
property depending on the feature selection. While performance properties can
be estimated without performance measurements, for instance by inferring
performance properties from software models \citep{woodside_future_2007},
measurement-based approaches for configurable systems address this optimization problem.  Proposed
approaches include performance prediction models build on learning performance behavior
with decision trees \citep{guo_variability-aware_2013}, learning a
frequency-based representation of the target function
\citep{zhang_performance_2015}, or learning the influence of single features and feature interactions
\citep{siegmund_predicting_2012,siegmund_performance-influence_2015}. All approaches have shown promising error rates for several real-world applications and allows prediction
of system performance for arbitrary configuration variants. However, all
approaches to create performance prediction models demand a samples of
performance measurements for multiple configurations to learn performance behavior and validate predictions.

\paragraph{Problem Statement}
The assessment of performance evolution requires a series of performance models
describing performance behavior for a series of versions of a configurable
system. Assessing the performance behavior for a single version of a
configurable software system entails a number of necessary and preliminary
tasks. These tasks can even become more complicated once instead of a single
version a series of versions is assessed:
\begin{itemize}
  \item \emph{Feature Model Synthesis:} Not all configurable systems do
  explicitly exhibit a variability model what is required to derive all valid variants
	\citep{rabkin_static_2011,nadi_where_2015}.
	While substantial work exists on reverse engineering variability models from
	source
	\citep{rabkin_static_2011,she_reverse_2011,zhou_extracting_2015,nadi_where_2015}
	code or non-code artifacts
	\citep{alves_exploratory_2008,andersen_efficient_2012,bakar_feature_2015}, many
	techniques still involve manual decisions \citep{she_reverse_2011} and domain
	knowledge \citep{nadi_where_2015}.
	Moreover, variability models evolve as part of the software
	\citep{peng_analyzing_2011}, vary from version to version, and, therefore,
	require repeated reverse engineering steps.
	
	\item \emph{Configuration Translation:} the translation of a valid
	configuration to a configuration artifact such as a configuration file or a list of command-line arguments may differ
from system to system. This step may be automated, but one still needs to
detect how configurations are read by the software system one wants to study.

\item \emph{Automated Integration:} Same goes for the infrastructure to
compile or build a software system since there exist may possible build tools such as makefiles or sbt.
Again, the build process can be automated, but one needs to detect and
specify the build mechanism used.

\item \emph{Version History Sampling:} To study performance evolution one
needs to specify which snapshots or versions of a software system one wants to study. While detecting
releases and release candidates should be straightforward, one might, for
instance, be interested in the performance evolution including snapshots
between two releases. As not all snapshots though are likely to compile,
classifying defect snapshots can still be tedious work.
\item \emph{Performance Assessment Setup:} The accurate assessment of
performance evolution requires a suitable testing setup. The methodology required for assessing performance
among others requires the selection of suitable performance metrics and
corresponding benchmarks, means to record measurements and repeat experiments
easily, and proper ways to interpret and compare results.
\end{itemize}

\paragraph{Goals And Thesis Structure}

The goal of this thesis is to provide a theoretical and practical foundation for
exhaustive performance measurements of configurable software systems and series
thereof. We contribute a guideline of and tool support for performance
measurements for configurable and evolving software systems. Our research
objectives and desired outcomes are

\begin{itemize}
\item a literature overview regarding software evolution, feature model
synthesis and performance assessment,
\item a methodology to assess performance evolution with respect to the
aforementioned challenges, and
\item a practical tool for performance measurement for multiple revisions of
configurable software systems.
\end{itemize}

The Thesis is organized as follows. Chapter \ref{chapter:2} provides the
background to the relevant topics discussed in this thesis, including
variability modeling, software evolution, the foundations and statistical
aspects of performance testing, and recent approaches to performance modeling. In
Chapter \ref{chapter:3}, we propose our measurement methodology and discuss
the methods used for our performance measurement tool as well as its
limitations. In Chapter \ref{chapter:4}, we evaluate several aspects of our
tool with respect to practicality and discuss the results thereof. Finally,
Chapter \ref{chapter:5} concludes the thesis and gives an outlook on possible
future work.
