Next, we conclude this thesis, review the limitations of this work, and provide
an outlook for future work on the topic of performance evolution of configurable
software systems.

\section{Concluding Remarks}
In this thesis, we have presented a methodology to assess the performance
evolution history of configurable software systems. Our methodology provides a
comprehensive and informed means to analyze performance of software systems
with respect to the dimensions of variability and evolution.

\paragraph{Methodology summary.} In the first part of our methodology,
chapter\,\ref{sec:chapter:3}, we provide an user guideline about how to
synthesize a variability model describing valid selections of configuration options. The guideline is driven by the extent of which variability is documented for the software system; for three different
scenarios (cf. Table\,\ref{tab:synthesis}), we advocate the use of existing
variability models, family-based analysis approaches, or a bottom-up strategy to pursue based on
documentation assets (cf. Table\,\ref{tab:manual_var_assessment}). 
The second part of our methodology in chapter\,\ref{chapter:4}, has
addressed the question of how to select a subset of versions of a configurable
software system to sketch performance evolution accurately and efficiently. We
have proposed and evaluated five different strategies based on assumed
relationships between software metrics and the impact of a revision on the
software system’s performance. The findings of our evaluation suggest that
versions, for which large code sections are revised, are a good approximation
for such version sample sets. Moreover, we were able to learn from larger revisions
the possible impact of modifying a file on the overall software system
performance. We have successfully tested the sampling strategies on a mature
configurable software system with a development history of about ten years.
Finally, in the third part of our methodology in chapter\,\ref{chapter:5}, we
have presented the practical aspects of performance measurement, including the selection of
suitable performance benchmarks, the choice of profiling tools, and statistical
means to summarize measurement results. In addition, we present  statistical
means to summarize and compare performance measurements across variants.

We have evaluated our methodology with a case study of two configurable software
systems (GNU XZ and x264). Using our methodology, for both software systems, we
were able to obtain a performance evolution history  of around a decade each.
We identified common patterns, among others a direct relationship between
performance degradation (execution time measurements increase) and revisions
indicating the introduction of new functionality. In addition, we observed
performance improvements for revisions indicating bug-fixes and refactorings.
However, both software systems exhibited different global trends regarding
their performance evolution history which suggests different levels of
maturity. As a more mature configurable software system, for x264, all tested
variants evolved homogeneously, while for GNU XZ, most tested variants evolved
heterogeneously and independent from each other. In fact, a subsequent
reliability assessment of the performance measurement tool indicated that
measurement spread was merely dependent on the software system tested, and was
lower for the more mature software system x264.

\paragraph{Research Contributions and Limitations.} We contribute a
methodological framework to obtain performance measurements for variant- and
version-rich software systems. This way, provide practitioners and the
research community an integrated and comprehensive set of guidelines for
analyzing and understand a configurable software system’s performance in two
dimensions, time and variability. Besides the methodological guidelines, we
provide the implementation of our experiment setup, used sampling strategies,
and performance measurement results for GNU XZ and x264 at
\url{http://www.github.com/smba/SPLPioneerPublic} to enable further research
and work in this field. As a further substantial contribution, we have proposed and
evaluated four revision sampling strategies to assess performance as close as
possible to the whole population of the performance of all revisions.
As we have learned from the evaluation of version sampling strategies as well
as the overall evaluation, the software system tested can have a substantial
impact on the methodology accuracy. Therefore, we push for more research that
is still required to render precisely the impact for software systems of
different levels of maturity and from different domains.

\section{Outlook and Future Work}
In the course of this thesis, several aspects arose that can be taken into
account for future work on the assessment of performance evolution for
configurable software systems. For the revision sampling strategy changed-files
sampling (cf. section\,\ref{sec:changedfiles}), we have learned and estimated
the impact of modifying a single file on the software system’s performance. Regarding this
sampling strategy driven by machine-learning, we propose the following possible
extensions that we believe sketch possible future directions. First, the
current features mapped to the performance influence are files. While the
sampling strategy with this level of granularity can be easily applied to
arbitrary software systems, more fine-grained feature-to-impact mappings are
possible. For instance, instead of files, functions or methods could be
conceived as features to map to performance impact estimates. Although this
requires additional  language-specific parsing, we believe this to be a
promising extension that might further increase the accuracy of this sampling
strategy. Second, the learned knowledge about the impact of modifying a
specific file (or function) can be used to localize those code sections that
are likely to impact performance. This knowledge, for instance, can be used to
advise future developers to be aware of the possible impact of modifying a
certain file. This might sketch a good basis for possible extensions to
integrated development environments.
